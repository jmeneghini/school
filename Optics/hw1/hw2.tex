\documentclass[a4paper]{article}

\usepackage{amsmath,amsfonts,amsthm,amssymb,mathtools} % Math packages
\usepackage[utf8]{inputenc} % Required for non-English characters
\usepackage[english]{babel} % Spell-checking
\usepackage{fancyhdr} % Required for custom headers
\usepackage{lastpage} % Required to determine the last page for the footer
\usepackage{extramarks} % Required for header and footer marks
\usepackage[margin=1.2in]{geometry} % Sets page margin
\usepackage{esdiff} % Defines commands for differentials
\mathtoolsset{showonlyrefs} % Only referenced equations are numbered
\usepackage{xfrac} % Allows for slanted fractions using \sfrac{*}{*}
\usepackage{graphicx} % For pictures
\usepackage{physics}
\usepackage{bm}

% The following sets up the header and footer
\fancyhf{}
\pagestyle{fancy}
\lhead{\hmwkCourse} %left header
\rhead{\hmwkTitle \, \textendash \, \hmwkName \, \textendash \, \hmwkClass} % right header
\rfoot{Page\ \thepage\ of\ \pageref{LastPage}} % right footer
\renewcommand\headrulewidth{0.4pt} % Size of the header rule
\renewcommand\footrulewidth{0.4pt} % Size of the footer rule

% The following sets the information shown in header and footer
\newcommand{\hmwkTitle}{Homework 2} % Assignment title
\newcommand{\hmwkCourse}{Optics} % Course/class
\newcommand{\hmwkName}{John Meneghini} % Your name
\newcommand{\hmwkClass}{2/9/2024}

% The following sets new paragraphs to start with a line skip instead of an indentation. Just my preference.
\setlength{\parindent}{0pt} %No indentation of new paragraphs
\setlength{\parskip}{10pt}

\begin{document}

\section*{Question 1}
Two thin lenses having focal lengths of $+15.0 cm$ and $-15.0$ cm respectively are positioned $60.0$ cm
apart. A page of print is held $25.0$ cm in front of the positive lens. Describe, in detail, the image of the
print (i.e. insofar as it’s paraxial). \\\\

\begin{center}
    $s_1 = 25.0$ cm, $f_1 = 15.0$ cm, $f_2 = -15.0$ cm, and $d = 60.0$ cm.
\end{center}
$$ \frac{1}{f_1} = \frac{1}{s_1} + \frac{1}{p_1} \rightarrow  \frac{1}{p_1} = \frac{1}{15.0} - \frac{1}{25.0}$$
$$ p_1 = 37.5 \textrm{ cm}$$
$$ \frac{1}{f_2} = \frac{1}{d - p_1} + \frac{1}{p_2} \rightarrow  \frac{1}{p_2} = -\frac{1}{15.0} - \frac{1}{60 - 37.5}$$
$$ p_2 = -9.0 \textrm{ cm}$$\\
What about magnification?
$$ m = m_1 m_2 = \left( - \frac{p_1}{s_1}\right) \left( - \frac{p_2}{s_2}\right)$$
$$ m = \left(-\frac{37.5}{25.0}\right) \left(\frac{9}{60-37.5}\right) = -0.5984$$
So, the image will be virtual, 9.0 cm in front of the negative lens (between the lenses), 0.5984 times smaller, and inverted.

\newpage
\section*{Question 2}
Draw a ray diagram for the combination of two positive lenses wherein their separation equals the
sum of their respective focal lengths. Do this carefully, with a ruler. You may do your sketch on the
arrangement shown on the last page of this assignment

\begin{figure*}[htb!]
    \centering
    \includegraphics[angle=90]{hw2q2.pdf}
\end{figure*}

\section*{Question 5}
Show that the triple scalar product $(\va{A} \cross \va{B}) \vdot \va{C}$ can be written as
$$ (\va{A} \cross \va{B}) \vdot \va{C} = 
    \begin{vmatrix}
    A_1 & A_2 & A_3 \\
    B_1 & B_2 & B_3 \\
    C_1 & C_2 & C_3 
    \end{vmatrix}  $$ \\\\

Write the vector's in term's of their components:

$$ \va{A} = \begin{bmatrix}
    A_1 \\
    A_2 \\
    A_3 \\
\end{bmatrix} $$
and similarly for $\va{B}$ and $\va{C}$.

So,
$$ \va{A} \cross \va{B} = \begin{bmatrix}
    A_1 \\
    A_2 \\
    A_3 \\
\end{bmatrix} \cross
\begin{bmatrix}
    B_1 \\
    B_2 \\
    B_3 \\
\end{bmatrix} = \begin{vmatrix}
    \vu{i} & \vu{j} & \vu{k} \\
    A_1 & A_2 & A_3 \\
    B_1 & B_2 & B_3 \\
    \end{vmatrix} = \vu{i}(A_2 B_3 - A_3 B_2) + \vu{j}(A_3 B_1 - A_1 B_3) + \vu{k}(A_1 B_2 - A_2 B_1) $$
by cofactor expansion of the determinant.
Then, 
$$(\va{A} \cross \va{B}) \vdot \va{C} = C_1 (A_2 B_3 - A_3 B_2) + C_2 (A_3 A_1 - A_1 A_3) + C_3 (A_1 A_2 - A_2 A_1)$$
We can reverse the cofactor expansion, giving
$$ (\va{A} \cross \va{B}) \vdot \va{C} = 
    \begin{vmatrix}
    C_1 & C_2 & C_3 \\
    A_1 & A_2 & A_3 \\
    B_1 & B_2 & B_3 \\
    \end{vmatrix}$$

Switching a row of the determinant switches the sign of the result:
$$ -(\va{A} \cross \va{B}) \vdot \va{C} = 
    \begin{vmatrix}
    A_1 & A_2 & A_3 \\
    C_1 & C_2 & C_3 \\
    B_1 & B_2 & B_3 \\
    \end{vmatrix}$$
Switching again, the sign once again becomes positive:
$$ (\va{A} \cross \va{B}) \vdot \va{C} = 
    \begin{vmatrix}
    A_1 & A_2 & A_3 \\
    B_1 & B_2 & B_3 \\
    C_1 & C_2 & C_3 
    \end{vmatrix}  $$ \\\\


\section*{Question 6}
Suppose we have a positive meniscus lens of radii 6 and 10 cm and a thickness of 3 cm, made with
a material of index of refraction $n = 1.5$. Determine its focal length and the locations of its principle
points.\\\\

Positive lens, so
\begin{center}
    $R_1 = 6$ cm, $R_2 = 10$ cm, $d = 3$ cm, and $n = 1.5$
\end{center}
First, we need $f$
$$ \frac{1}{f} = (n - 1) \left(\frac{1}{R_1} - \frac{1}{R_2} + \frac{(n-1)d}{nR_1 R_2}\right) = (1.5 - 1) \left(\frac{1}{6} - \frac{1}{10} + \frac{(1.5-1)(3)}{(1.5)(6)(10)}\right)$$
$$ f = 25 \textrm{ cm} $$
$$ h_1 = - \frac{f(n-1)d}{nR_2} = - \frac{(25)(1.5-1)(3)}{(1.5)(10)} = -2.5 \textrm{ cm}$$
$$ h_2 = - \frac{f(n-1)d}{nR_1} = - \frac{(25)(1.5-1)(3)}{(1.5)(6)} = -4.1667 \textrm{ cm}$$

\section*{Question 7}
A spherical glass bottle 20 cm in diameter with walls that are negligibly thin is filled with water.
The bottle is sitting on the back seat of a car on a nice, sunny day. What is the focal length of the
“lens?”\\\\

\begin{center}
    $R_1=10$ cm, $R_2=-10$ cm, $d = 20$ cm, and $n = 1.333$
\end{center}
$$ \frac{1}{f} = (n - 1) \left(\frac{1}{R_1} - \frac{1}{R_2} + \frac{(n-1)d}{nR_1 R_2}\right) = (1.333 - 1) \left(\frac{1}{10} + \frac{1}{10} + \frac{(1.333-1)(20)}{(1.333)(10)(-10)}\right)$$
$$ f = 20 \textrm{ cm}$$


\section*{Question 8}
It is found that sunlight is focused to a spot 29.6 cm from the back face of a thick lens, which has
principle points $h_1 = 0.2$ cm and $h_2 = -0.4$ cm. Determine the location of the image of a candle that
is placed 49.8 cm in front of this lens.\\\\

Assuming the rays of light from the sun are parallel, this means the $s \rightarrow \infty$ and $p=f$, but $f$ and $p$ are measured from $h_2$, making
$p = 29.6 + 0.4 = 30.0$ cm $=f$. So,
$$ \frac{1}{s} + \frac{1}{p} = \frac{1}{f} \rightarrow \frac{1}{p} = \frac{1}{30} - \frac{1}{s}$$
But, $s$ is measured from the $h_1$, so $s = 49.8 + 0.2 = 50.0$ cm.
$$ \frac{1}{p} = \frac{1}{30} - \frac{1}{50.0} \rightarrow p=75.0 \textrm{ cm}$$
So, the image will be located $75.0 - 0.4=74.6$ cm from the back of the lens.

\section*{Question 9}
A crown glass double-convex lens that is 4.0 cm thick has an index of refraction of 3/2. Given that
its radii are 4.0 cm and 15 cm, locate its principle points and compute its focal length. If a television
screen is placed 1.0 m from the front of the lens, where will the real image of the picture appear?\\\\

Since the lens is double-convex, $R_2$ is negative.
\begin{center}
    $R_1 = 4.0$ cm, $R_2 = -15$ cm, $d = 4.0$ cm, and $n = 3/2$
\end{center}
$$ \frac{1}{f} = (n - 1) \left(\frac{1}{R_1} - \frac{1}{R_2} + \frac{(n-1)d}{nR_1 R_2}\right) = (1.5 - 1) \left(\frac{1}{4} + \frac{1}{15} + \frac{(1.5-1)(4)}{(1.5)(4)(-15)}\right)$$
$$ f = 6.79 $$
$$ h_1 = - \frac{f(n-1)d}{nR_2} = - \frac{(6.79)(1.5-1)(4)}{(1.5)(-15)} = 0.604 \textrm{ cm}$$
$$ h_2 = - \frac{f(n-1)d}{nR_1} = - \frac{(6.79)(1.5-1)(4)}{(1.5)(4)} = -2.264 \textrm{ cm}$$
$$ \frac{1}{s} + \frac{1}{p} = \frac{1}{f} \rightarrow \frac{1}{p} = \frac{1}{6.79} - \frac{1}{1.0 + 0.604}$$
$$ p = -2.1 \textrm{ cm}$$
But $p$ is measured from $h_2$, so the image is $-2.264 - 2.1 = -4.36$ cm from the back of the lens, which is 0.36 cm in front of the lens.

\section*{Question 10}
Given the following matrices,

$$ 
\bm{A} = \begin{bmatrix}
    1 & 2 & -1 \\
    0 & 3 & 1 \\
    2 & 0 & 1 \\
\end{bmatrix},\;
\bm{B} = \begin{bmatrix}
    2 & 1 & 0 \\
    0 & -1 & 2 \\
    1 & 1 & 3 \\
\end{bmatrix} $$
work out the product $\bm{A}\bm{B}$.\\\\


$$\bm{A}\bm{B} = \begin{bmatrix}
    1 & 2 & -1 \\
    0 & 3 & 1 \\
    2 & 0 & 1 \\
\end{bmatrix}
\begin{bmatrix}
    2 & 1 & 0 \\
    0 & -1 & 2 \\
    1 & 1 & 3 \\
\end{bmatrix}$$
$$ = \begin{bmatrix}
    (1)(2) + (2)(0) + (-1)(1) & (1)(1) + (2)(-1) + (-1)(1) & (1)(0) + (2)(2) + (-1)(3) \\
    (0)(2) + (3)(0) + (1)(1) & (0)(1) + (3)(-1) + (1)(1) & (0)(0) + (3)(2) + (1)(3) \\
    (2)(2) + (0)(0) + (1)(1) & (2)(1) + (0)(-1) + (1)(1) & (2)(0) + (0)(2) + (1)(3) \\
\end{bmatrix} $$
$$ = \begin{bmatrix}
    1 & -2 & 1 \\
    1 & -2 & 9 \\
    5 & 3 & 3 \\
\end{bmatrix}$$





\end{document}